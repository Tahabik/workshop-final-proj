\documentclass{article}

\usepackage[utf8]{inputenc}
\usepackage{geometry}
\geometry{a4paper, margin=1in}
\usepackage{graphicx}
\usepackage{hyperref}
\usepackage{fancyhdr}

\pagestyle{fancy}
\fancyhf{} 


\fancyhead[L]{\Large Final Assignment}


\fancyhead[R]{\Large Computer Workshop}


\renewcommand{\headrulewidth}{1pt}


\title{\Huge The Final Assignment \\[1cm]
\Large Author's Name: Taha Biklaryan \\[0.5cm]
\large Student ID: 401411318 \\[1cm]  
\large Dr. Malekimajd \\[1cm]
\large Computer Workshop}
\date{} % Remove date

\begin{document}

\maketitle
\newpage
\section*{\Huge 1 Git and Github}
\subsection*{\Large 1.1 Repository Initialization and Commits}
For creating a repository for this project, first i opened my account and then in the repository section of the github i created a repository and named it "Workshop-Final-proj" and then 
cloned the repository into a local directory of my pc and synced it with my vscode where i'm editing this assignment right now and going to commit the changes so far right now after putting the dot
at the end of this sentence.\\
\subsection*{\large 1.2 Github Actions for LaTeX Compilation}
After setting up the repository in github i created a main.tex that i'm constantly commiting changes there
and i created folder .github which is workflow to i build and release my LaTeX files through github Actions.
And after each changes in the text editor i commit the changes to the main.tex file via vscode.\\


\section*{\Huge 2 Exploration Tasks}
\subsection*{\large 2.1 Vim Advanced Features}
1. Sessions:
Vim sessions allow you to save and restore your working environment, including open files, cursor positions, and window layouts. You can create a session with ':mksession' and then later restore it using ':source Session.vim'. This is particularly useful when you are working on a project and want to quickly resume your work with the same setup.\\
2. Marks And Jumps:
Vim allows you to place marks in your file, which act as bookmarks. You can then jump between these marks using various commands. For example, you can use 'ma' to mark a position with the label 'a', and then use ''a' to jump back to that position. This is handy for navigating large files and remembering specific locations.\\
3. Text Objects:
Vim provides a variety of text objects that allow you to operate on different parts of text efficiently. For instance, 'iw' represents "inner word," and 'i(' represents "inner parentheses." These text objects can be combined with commands to perform powerful operations on specific chunks of text.\\
\subsection*{\large 2.2 Memory Profiling}


\end{document}
